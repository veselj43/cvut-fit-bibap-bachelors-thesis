% options:
% thesis=B bachelor's thesis
% thesis=M master's thesis
% czech thesis in Czech language
% slovak thesis in Slovak language
% english thesis in English language
% hidelinks remove colour boxes around hyperlinks

\documentclass[thesis=B,czech]{FITthesis}[2012/06/26]

\usepackage[utf8]{inputenc} % LaTeX source encoded as UTF-8

\usepackage{graphicx} %graphics files inclusion
% \usepackage{amsmath} %advanced maths
% \usepackage{amssymb} %additional math symbols

\usepackage{dirtree} %directory tree visualisation

% % list of acronyms
% \usepackage[acronym,nonumberlist,toc,numberedsection=autolabel]{glossaries}
% \iflanguage{czech}{\renewcommand*{\acronymname}{Seznam pou{\v z}it{\' y}ch zkratek}}{}
% \makeglossaries

\newcommand{\tg}{\mathop{\mathrm{tg}}} %cesky tangens
\newcommand{\cotg}{\mathop{\mathrm{cotg}}} %cesky cotangens

% % % % % % % % % % % % % % % % % % % % % % % % % % % % % % 
% ODTUD DAL VSE ZMENTE
% % % % % % % % % % % % % % % % % % % % % % % % % % % % % % 

\department{Katedra softwarového inženýrství}
\title{Systém pro skórování Ultimate Frisbee zápasů - frontend}
\authorGN{Jaroslav} %(křestní) jméno (jména) autora
\authorFN{Veselý} %příjmení autora
\authorWithDegrees{Jaroslav Veselý} %jméno autora včetně současných akademických titulů
\supervisor{Ing. Jiří Hunka}
%\acknowledgements{Doplňte, máte-li komu a za co děkovat. V~opačném případě úplně odstraňte tento příkaz.}
\abstractCS{V~několika větách shrňte obsah a přínos této práce v~češtině. Po přečtení abstraktu by se čtenář měl mít čtenář dost informací pro rozhodnutí, zda chce Vaši práci číst.}
\abstractEN{Sem doplňte ekvivalent abstraktu Vaší práce v~angličtině.}
\placeForDeclarationOfAuthenticity{V~Praze}
\declarationOfAuthenticityOption{4} %volba Prohlášení (číslo 1-6)
\keywordsCS{frontend, ultimate frisbee, uživatelské rozhraní}
\keywordsEN{frontend, ultimate frisbee, user interface}

\begin{document}

% \newacronym{CVUT}{{\v C}VUT}{{\v C}esk{\' e} vysok{\' e} u{\v c}en{\' i} technick{\' e} v Praze}
% \newacronym{FIT}{FIT}{Fakulta informa{\v c}n{\' i}ch technologi{\' i}}

\begin{introduction}

	Cílem bakalářské práce je implementace frontendu pro webovou aplikaci umožňující skórování frisbee a následné poskytování statistik. Aplikace je vyvíjena v rámci dvou souběžných bakalářských prací, tato je zaměřena na frontend a uživatelské rozhraní, druhá na backend.

	\section{Ultimate frisbee}

		Ultimate frisbee je sport, ve kterém soutěží vždy dva týmy proti sobě a vítězí tým s vyšším počtem bodů. Na rozdíl od ostatních sportů je ale Utimate Frisbee přímo založené na sportovním duchu. Kromě klasického bodování průběhu hry se ještě hodnotí Spirit of the Game. Hráči tedy po každém zápase hodnotí férovost soupeře. V každém turnaji se vyhlašuje i cena Spirit of the game, která je ceněna obdobně jako cena za první místo. Díky tomuto přístupu mají odpovědnost za férovost hry samotní hráči a není zde zapotřebí rozhodčí \cite{frisbee}.

		Hodnocení pomocí aplikace je zde ještě větším přínosem než u většiny sportů právě kvůli hodnocení férovosti (sportovního ducha) soupeře. Toho hodnocení se jinak vyplňuje na papírovém dotazníku, které je zapotřebí zpracovat. Aplikace by tuto práci měla značně usnadnit.

	\section{Motivace}

		Realizace této práce vzešla z nápadu na rozšíření mobilní aplikace Catcher, kterou čeští hráči frisbee používají. 
		Jeden z požadavků byl, aby se aplikace dala použít na různých zařízeních, jelikož pro organizaci turnaje je pohodlnější použít notebook či stolní počítač, kdežto pro získání informací o zápasech, turnajích a hodnocení spirit of the game může být praktičtější mobilní zařízení. Další z~možných rozšíření by bylo například propojení s databází České asociace létajícího disku.

\end{introduction}

\chapter{Analýza}
	% TODO neco jsem zde chtel jeste uvest

	\section{Proč nová aplikace}
		Než se začnu zabývat analýzou, chtěl bych ještě zdůvodnit upřednostnění vývoje nové aplikace před upravením původní.

		Aplikace Catcher byla vyvinuta jako mobilní aplikace za účelem jednoduššího skórování frisbee ultimate predevším pro soutěže ČALD. Slouží ke skórování za pomoci mobilního zařízení s OS Android a následném zpřístupnění výsledků na webu. Co se týče implementace, backend aplikace nemá žádné standardizované API, které by bylo možné upravit.

		Z toho pro nás vyplynulo několik důvodů proč bychom chtěli vytvořit svou vlastní implementaci jak backendu tak frontendu. První důvod spočívá v komunikaci mezi backendem a frontendem, zde bychom chteli mít jasně daný styl komunikace (REST API), aby bylo možná implementace frontendu na různých platformách. Přepracování stávající aplikace by již z tohoto ohledu bylo pracné s ohledem na fakt, že bychom nejprve museli porozumnět cizímu kódu. Dalším důvodem je již zmiňované použití na různých zařízeních. Při rozhodnutí pro úpravu bychom museli upravit frontend aplikace, čímž by se zachovala pouze stávající mobilní aplikace a následně by bylo zapotřebí implementovat buďto desktopovou nebo webovou aplikaci pro pokrytí požadavků, které bychom chtěli splnit. Místo toho lze implementovat webovou službu jako backend (REST API) a webovou aplikaci jako frontend. Problém zobrazení na různých zařízeních lze alespoň dočasně řešit responzibilitou webu a pokud se aplikace osvědší v praxi, lze implementovat aplikace lépe optimalizované jednotlivým typům zařízení.
	
	\section{Problematika}
		% TODO klasicky problem predelat vs nova aplikace ?

		% jak se situace má nyní a co by mohlo být lepší

		% ... Vzhledem k tématu aplikace je několik funkčních požadavků
	\section{Vztah uživatele s aplikací}
		Již z tématu jako takového vyplývá několik scénářů a použití.

		Lidé, kteří se o tento sport zajímají se dají rozdelit do následujících skpin:

		\begin{itemize}
			\item Organizátoři
			\item Zástupci týmů (trenéři)
			\item Hráči
			\item Fanoušci
		\end{itemize}

		Pro běh aplikace budou zapotřebí tyto role:

		\begin{itemize}
			\item Správce
			\item Organizátor
			\item Tým
			\item Fanoušek
		\end{itemize}

		\subsection{Správce}

	% TODO posat klíčové akce, které budou uživatelé provádět

	% TODO posat další funkcionalitu aplikace

	% TODO persony - ui pohled

	\section{Možnosti řešení}
		Systému pro skórování Ultimate frisbee by mohl být provozován jako mobilní, desktopová nebo webová aplikace. Vzhledem k použitelnosti aplikace je potřeba data ukládat na server, aby nebylo celé hodnocení závislé pouze na jediném zařízení. Možnost importu a exportu dat je zde nedostačující, v průběhu turnaje mají mít všichni přehled a není vhodné čekat na export dat.
		
		\subsection{Mobilní aplikace}
			Mobilní aplikace má výhodu dostupnosti. Nevýhodou je psaní rozpisů týmů, hráčů a celková tvorba turnajů. Je nepraktické psát takové množství dat na mobilním telefonu či tabletu a to i v případě, že by existovala databáze hráčů. Tento problém by částečně řešila možnost importu dat, zde bylo zas bylo složité přesně specifikovat očekávaný formát a to hlavně u rozpisu turnajů.
		
		\subsection{Desktopová aplikace}
			Desktopová aplikace je na rozdíl od mobilní vhodná pro vyplňování dat, ale nepraktická při hodnocení sportovního ducha soupeře. Turnaje se konají i na venkovních hřištích a každý tým by tak musel mít k dispozici notebook. Zkrátka by zde chyběla požadovaná mobilita.
		
		\subsection{Webová aplikace}
			Webová aplikace má značnou výhodu v přenosnosti mezi platformami, data lze vyplnit na jakémkoli PC a hodnotit lze na mobilním zařízení. Optimalizace pro oba typy zařízení by zde měla být dostačující.
		
		\subsection{Způsob propojení backendu a frontendu}
			Při tvorbě aplikace se nabízí možnost vytvořit API, pomocí kterého bude komunikovat serverová část aplikace starající se o data s uživatelským rozhraním, které bude poskytovat možnosti práce s těmito daty. Jednoznačnou výhodou je zde možnost další implementace více druhů aplikací, které budou optimalizované pro konkrétní použití a budou sdílet data. Tento přístup by měl zajistit možnost rozšiřování jak z hlediska funkcionality tak z hlediska vývoje aplikací pro libovolné zařízení, které má možnost připojení k internetu.
		
	\section{Zvolené řešení}
		S ohledem na celkem rozhodující faktor pro použitelnost aplikace v praxi, kterým je dostupnost jsme zvolili webovou aplikaci s tím, že komunikace zařízení a serveru bude probíhat přes Rest API. % TODO
		
		V praxi by mohl turnaj probíhat jako příprava turnaje předem na desktopovém počítači nebo notebooku. Jeden notebook by byl k~dispozici pro bodování průběhu hry. Dále by měl každý tým nějaké mobilní zařízení, na kterém by hodnotil spirit of the game. Tato funkce je zároveň jednou z velice užitečných oproti skórování bez aplikace, kde týmy vyplňují dotazníky, které se musí následně ručně zpracovávat a vyhodnocovat.

	% TODO použité technologie

\chapter{Návrh aplikace}
	Doposud jsem vymezil prostřední, ve kterém se aplikace bude vyvýjet, nyní se zaměřím na nástroje, které k vývoji použiji.

	% github

	% TODO 
	\section{Zvolené technologie}
		Z analýzy vyplynulo, že budu navrhovat webovou aplikaci. Použití HTML a CSS je tedy automatické. Otázkou zůstává jaký programovací jazyk použít. Jelikož by se melo jednat o moderní uživatelské rozhraní, bude potřeba interaktivních prvků, tedy použití javascriptu. Nyní je k dispozici několik možností jak komunikovat přes API a jak pracovat se získanými daty. Možností je mnoho, lze použít například PHP, python nebo zůstat rovnou u javascriptu. Vzhledem k interaktivitě a k dnešním možnostem javascriptu se mi tato možnost zamlouvá, ale zbývá zhodnotit, zda nemá i kritické slabiny, které by chod aplikace mohly ovlivnit. Možnou nevýhodou JS je práce s daty na uživatelově zařízení, je tedy možnost, že by mohl uživatel upravit data tak, že by to narušilo chod aplikace? A to jak chtěně tak i nechtěně? O data se bude starat backend, kde se data mění jen na základě jednotlivých požadavků. Každý požadavek je vždy zkontrolován na straně serveru, což uživatel ovlivnit nemůže. Pokud tedy uživatel upraví data pouze na svém zařízení, má to vliv pouze na uživatele, který tak učinil a to jen než například obnoví navštěvovanou stránku.

		Javascript je dnes velmi populární, existuje mnoho knihoven a frameworků. Rozhodl jsem se toho využít a vybral jsem si AngularJS. Co se stylů týče, použiji bootstrap, který umožní jednoduché a responzivní umístění prvků.

		\subsection{Bootstrap}
		
		\subsection{AngularJS}

	\section{Použitá řešení}

\chapter{Realizace}

\chapter{Testování uživatelského rozhraní}
	
	\section{Heuristická analýza}
		Tato analýza spočívá v kontrole uživatelského rozhraní expertem, resp. někým z oboru, oproti obecně osvědčeným a ověřeným pravidlům či postupům. 
	% TODO vzdy pojmenovat problem a priradit dulezitost

		\subsection{Viditelnost stavu systému}
			Aplikace celkově neobsahuje příliš úrovní obsahu. Uživatel vždy vidí zvýrazněnou položku v menu. Pokud se některá akce skládá z více kroků nebo je k dispozici více možností, je k dispozici orientace pomocí záložek. Běžný uživatel se setká maximálně se dvěma úrovněmi menu a to hlavní menu a záložkové. V administraci jsou i případy, kde je použita ještě třetí úroveň, jednotlivé úrovně jsou přehledně odsazené, takže by se uživatel neměl v obsahu ztácet.

			Bylo by dobré ještě doplnit uživatelský panel po přihlášení, nyní je k dispozici jen tlačítko pro odhlášení. Informace o roli uživatele je jen ve zprávě o úspěšném přihlášení, poté již není tato informace uživateli dostupná.

			Uživatel je o výsledku každé akce informován pomocí zprávy, která je stylovaná, tedy i barevně odlišená podle obsahu, bootstrapem.

		\subsection{Propojení systému s reálným světem}
			Neměly by být použité žádné termíny neznámé uživateli, některá slova jsou převzata z prostření frisbee, jinak je použitá jen běžná slovní zásoba.

		\subsection{Uživatelova kontrola a svoboda}
			Celá aplikace podporuje tlačítka "zpět" a "dopředu". Při vícekrokovém procesu je vždy možnost vrátit se zpět také pomocí již víše zmiňovaných záložek.

		\subsection{Předcházení chybám}
			U formlulářů je u každé položky poskytnuta informace o chybě, pokud je některá z podmínek pro formát vsupu porušena. Při odeslání údajů se zkontrolují případné další chyby o kterých je uživatel informován, přičemž nepřichází o žádná vyplněná data.

		\subsection{Připomenutí raději než odvolávání se}
			Některé složitější formuláře interagují s vyplněnými daty například z předchozích kroků. Data jako taková uživatel žadává jen jednou, poté již pouze vybírá ze seznamu možností.

		\subsection{Flexibilita a efektivita použití}
			Prozatím není aplikace příliš rozsáhlá a není zapotřebí mnoho kroků pro dokončení libovolné akce, navíc jsou všechny kroky povinné, pouze minimum informací je doplňujícího či popisujícího charakteru.

		\subsection{Estetický a minimalisticý design}
			Každá stránka obsahuje pouze informace k dané akci a informace k tomu potřebné, jakékoli zprávy poskytované uživateli jsou shrnuté do stručné věty, popřípadě několika málo vět vystihujících dění v aplikaci.

		\subsection{Pomoci uživatelům rozpoznat, diagnostikovat a zotavit se z chyb}
			Chybové hlášky obsahujý nejdříve prostý text popisující chybu běžnému uživateli, pokud je chyba způsobena systémem je o tom uživatel ve zprávě informován. Jestliže jsou k dispozici další poskytnutelné upřesňující informace o chybě, jsou zahrnuty ve zprávě z důvodu rozpoznání chyby vývojáři.

		\subsection{Pomoc a dokumentace}
			% TODO

	\section{Kognitivní průchod}

	\section{Testování použitelnosti}

\begin{conclusion}
	%sem napište závěr Vaší práce
\end{conclusion}

\bibliographystyle{csn690}
\bibliography{mybibliographyfile}

\appendix

\chapter{Seznam použitých zkratek}
% \printglossaries
\begin{description}
	\item[UI] User interface
	\item[HTML] HyperText markup language
	\item[CSS] Cascading style sheets
	\item[ČALD] Česká asociace létajícího disku
\end{description}


% % % % % % % % % % % % % % % % % % % % % % % % % % % % 
% % Tuto kapitolu z výsledné práce ODSTRAŇTE.
% % % % % % % % % % % % % % % % % % % % % % % % % % % % 
% 
% \chapter{Návod k~použití této šablony}
% 
% Tento dokument slouží jako základ pro napsání závěrečné práce na Fakultě informačních technologií ČVUT v~Praze.
% 
% \section{Použití šablony}
% 
% Šablona je určena pro zpracování systémem \LaTeXe{}. Text je možné psát v~textovém editoru jako prostý text, lze však také využít specializovaný editor pro \LaTeX{}, např. Kile.
% 
% Pro získání tisknutelného výstupu z~takto vytvořeného souboru použijte příkaz \verb|pdflatex|, kterému předáte cestu k~souboru jako parametr. Vhodný editor pro \LaTeX{} toto udělá za Vás. \verb|pdfcslatex| ani \verb|cslatex| \emph{nebudou} s~těmito šablonami fungovat.
% 
% Více informací o~použití systému \LaTeX{} najdete např. v~\cite{wikilatex}.
% 
% \subsection{Typografie}
% 
% Při psaní dodržujte typografické konvence zvoleného jazyka. České \uv{uvozovky} zapisujte použitím příkazu \verb|\uv|, kterému v~parametru předáte text, jenž má být v~uvozovkách. Anglické otevírací uvozovky se v~\LaTeX{}u zadávají jako dva zpětné apostrofy, uzavírací uvozovky jako dva apostrofy. Často chybně uváděný symbol "{} (palce) nemá s~uvozovkami nic společného.
% 
% Dále je třeba zabránit zalomení řádky mezi některými slovy, v~češtině např. za jednopísmennými předložkami a spojkami (vyjma \uv{a}). To docílíte vložením pružné nezalomitelné mezery -- znakem \texttt{\textasciitilde}. V~tomto případě to není třeba dělat ručně, lze použít program \verb|vlna|.
% 
% Více o~typografii viz \cite{kobltypo}.
% 
% \subsection{Obrázky}
% 
% Pro umožnění vkládání obrázků je vhodné použít balíček \verb|graphicx|, samotné vložení se provede příkazem \verb|\includegraphics|. Takto je možné vkládat obrázky ve formátu PDF, PNG a JPEG jestliže používáte pdf\LaTeX{} nebo ve formátu EPS jestliže používáte \LaTeX{}. Doporučujeme preferovat vektorové obrázky před rastrovými (vyjma fotografií).
% 
% \subsubsection{Získání vhodného formátu}
% 
% Pro získání vektorových formátů PDF nebo EPS z~jiných lze použít některý z~vektorových grafických editorů. Pro převod rastrového obrázku na vektorový lze použít rasterizaci, kterou mnohé editory zvládají (např. Inkscape). Pro konverze lze použít též nástroje pro dávkové zpracování běžně dodávané s~\LaTeX{}em, např. \verb|epstopdf|.
% 
% \subsubsection{Plovoucí prostředí}
% 
% Příkazem \verb|\includegraphics| lze obrázky vkládat přímo, doporučujeme však použít plovoucí prostředí, konkrétně \verb|figure|. Například obrázek \ref{fig:float} byl vložen tímto způsobem. Vůbec přitom nevadí, když je obrázek umístěn jinde, než bylo původně zamýšleno -- je tomu tak hlavně kvůli dodržení typografických konvencí. Namísto vynucování konkrétní pozice obrázku doporučujeme používat odkazování z~textu (dvojice příkazů \verb|\label| a \verb|\ref|).
% 
% \begin{figure}\centering
% 	\includegraphics[width=0.5\textwidth, angle=30]{cvut-logo-bw}
% 	\caption[Příklad obrázku]{Ukázkový obrázek v~plovoucím prostředí}\label{fig:float}
% \end{figure}
% 
% \subsubsection{Verze obrázků}
% 
% % Gnuplot BW i barevně
% Může se hodit mít více verzí stejného obrázku, např. pro barevný či černobílý tisk a nebo pro prezentaci. S~pomocí některých nástrojů na generování grafiky je to snadné.
% 
% Máte-li například graf vytvořený v programu Gnuplot, můžete jeho černobílou variantu (viz obr. \ref{fig:gnuplot-bw}) vytvořit parametrem \verb|monochrome dashed| příkazu \verb|set term|. Barevnou variantu (viz obr. \ref{fig:gnuplot-col}) vhodnou na prezentace lze vytvořit parametrem \verb|colour solid|.
% 
% \begin{figure}\centering
% 	\includegraphics{gnuplot-bw}
% 	\caption{Černobílá varianta obrázku generovaného programem Gnuplot}\label{fig:gnuplot-bw}
% \end{figure}
% 
% \begin{figure}\centering
% 	\includegraphics{gnuplot-col}
% 	\caption{Barevná varianta obrázku generovaného programem Gnuplot}\label{fig:gnuplot-col}
% \end{figure}
% 
% 
% \subsection{Tabulky}
% 
% Tabulky lze zadávat různě, např. v~prostředí \verb|tabular|, avšak pro jejich vkládání platí to samé, co pro obrázky -- použijte plovoucí prostředí, v~tomto případě \verb|table|. Například tabulka \ref{tab:matematika} byla vložena tímto způsobem.
% 
% \begin{table}\centering
% 	\caption[Příklad tabulky]{Zadávání matematiky}\label{tab:matematika}
% 	\begin{tabular}{|l|l|c|c|}\hline
% 		Typ		& Prostředí		& \LaTeX{}ovská zkratka	& \TeX{}ovská zkratka	\tabularnewline \hline \hline
% 		Text		& \verb|math|		& \verb|\(...\)|	& \verb|$...$|		\tabularnewline \hline
% 		Displayed	& \verb|displaymath|	& \verb|\[...\]|	& \verb|$$...$$|	\tabularnewline \hline
% 	\end{tabular}
% \end{table}
% 
% % % % % % % % % % % % % % % % % % % % % % % % % % % % 

\chapter{Obsah přiloženého CD}

%upravte podle skutecnosti

\begin{figure}
	\dirtree{%
		.1 readme.txt\DTcomment{stručný popis obsahu CD}.
		.1 exe\DTcomment{adresář se spustitelnou formou implementace}.
		.1 src.
		.2 impl\DTcomment{zdrojové kódy implementace}.
		.2 thesis\DTcomment{zdrojová forma práce ve formátu \LaTeX{}}.
		.1 text\DTcomment{text práce}.
		.2 thesis.pdf\DTcomment{text práce ve formátu PDF}.
		.2 thesis.ps\DTcomment{text práce ve formátu PS}.
	}
\end{figure}

\end{document}
